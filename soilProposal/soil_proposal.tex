\documentclass[12pt]{article}
\usepackage{fullpage,enumitem,amsmath,amssymb,graphicx, listings,cite}%,hyperref}

\usepackage{hyperref}
\usepackage{xcolor}
\hypersetup{
   colorlinks,
   linkcolor={red!50!black},
   citecolor={blue!50!black},
   urlcolor={blue!80!black}
}

\title{Enabling Affordable Precision Agriculture by Sensing Soil Moisture Wirelessly}
\author{Colleen Josephson\\Advised by:  Ranveer Chandra (Microsoft) and Sachin Katti (Stanford)\\\textbf{Do not forward}}

\begin{document}
\maketitle

\begin{abstract}
  The term ``precision agriculture'' has been around since the 1980s,
  and it describes a farming technique that uses extensive measurement
  of crop data to make informed decisions about watering,
  fertilization and more. Despite decades of work, precision
  agriculture is still not widely implemented on working farms. This
  is because the data is hard to collect and process, primarily due to
  the lack of high speed internet access in rural areas, and the high
  cost and difficulty of deploying dense sensor networks. Recent work
  such as Farmbeats~\cite{farmbeats} has made good progress in
  addressing the issues of internet access, but the cost and
  deployability of sensors remains an issue. We seek to address this
  by designing a low-cost and easy-to-use wireless soil moisture
  sensor.
\end{abstract}


\section*{Introduction}

Agriculture is the single largest pressure on the world's sources of
fresh water--- 69\% of the global fresh water supply is used for
agriculture~\cite{water}. Paired with the fact that the world's
population is projected to exceed 9 billion by 2050~\cite{population},
with most of the growth coming from developing nations in Africa,
conservation of fresh water and sufficient food production are key
concerns that need to be addressed. Soil moisture is the most
important measurement for ensuring the maximization of crop yield
without water waste.

Multiple studies show that soil moisture sensors lead to a water
savings of at least 20\%\cite{watersavings}, and in some cases more
than 50\%, while maintaining crop yields. Yet soil sensors are still
not widely deployed on working farms, despite decades of research
confirming the benefits. The lack of widespread adoption can be
attributed to a few key challenges: 1.) high sensor cost 2.)
difficulty of deploying and maintaining the sensors and 3.) difficulty
collecting and processing the sensor data.

The average commercial\footnote{People are often misled by the
  SparkFun soil moisture sensor retailing for \$6.95. This is just a
  raw sensor probe, with no wiring, weatherproofing, testing or
  calibration. The added cost of a power source, data logger and
  wiring is more than \$40, and the system is still not weatherproofed
  or calibrated.} soil moisture sensor is more than \$100, which does
not include the power source or data logger to record and/or transmit
the measurement samples. The cost is at least \$50 to add the cheapest
DIY weatherproof logger and transmitter.  Since soil is not uniform
across a field, nevermind an entire farm, multiple sensors are needed
to accurately measure moisture for irrigation purposes. The average
farm in the United States is 444 acres~\cite{farmsize}. The water cost
for a tomato farm of that size would be approximately \$55,000 per
year~\cite{agwater}~\cite{tomatoes}, which means a precision soil
moisture system would lead to \$11,000 in savings. However, the cost
of deploying 20 sensors~\cite{sensorDensity} per acre would more than
\$1,000,000 with today's sensor costs. The current high cost of
sensors makes it difficult to for the average farmer to justify
investing in precision agriculture. Furthermore, this cost makes soil
moisture sensing completely infeasible for small-holder farmers in
developing nations, which is where the most of the food and water
insecurity will be located.

On top of the high cost, current sensors are not simple to deploy and
maintain. Very few sellers offer a product that includes the sensor,
logger and power source ready for immediate use. Therefore some amount
of setup labor is required for each sensor. Figure 1 shows a typical
sensor and data logging system. Though the sensor is waterproof, the
data logger (in this case an Arduino) needs to be waterproofed and
powered. The soil moisture sensor needs to be buried, which requires
digging a hole to the desired depth. This project also attaches solar
panel to the battery pack, which requires mounting high enough to
harvest sunlight, probably requiring a wooden or metal post. The
laborious process of burying the sensor and mounting the solar panel
would need to be repeated for every sensor node on the
farm. Furthermore, the excess cables and bulky box also make the
system easy to tangle in farm equipment and tools. The sensor system
would need to gently removed and re-deployed every time the field is
tilled. This all adds up to a significant amount of manual labor to
deploy and maintain the soil moisture sensors.

\begin{figure*}[h!]
  \centering
  \includegraphics[scale=0.75]{soil_moisture_sn_setup.jpg}\\
  \caption{A weatherproofed soil moisture sensor box from the
    Geosensor Networks Lab~\cite{sensorbox}}
\end{figure*}

Finally, the data needs to be collected from the logger. For a 444
acre farm, the most practical method is wireless collection. Extending
wireless coverage to a large farm is not simple, especially because
cellular coverage in rural areas tends to be poor. A number of recent
works have considered the issue of networking sensors in rural
environments. For example, the 2015 Farmbeats project looks at the
issue of sensor data collection and network access extensively, using
TV whitespace technology to provide a wireless gateway from the field
to the Internet. Though solving this challenge is also important, the
focus of our research will be on reducing the cost and improving the
deployability and maintainability of soil moisture sensors.

In the next section we will give an overview of current sensing
technology and related work.

\section*{Overview of sensing techniques}

The most accurate way to measure soil moisture is gravimetrically,
where a soil sample is taken, weighed, allowed to dry (by oven or
otherwise) and re-weighed~\cite{Noborio2001}. This process is
time-intensive and requires physical removal of soil at the depth you
wish to measure, so it is impractical for irrigation purposes. There
are three primary types of practical soil moisture sensors used in
agriculture: tensiometers, resistive sensors, and volumetric
sensors~\cite{sensorTypes}.

\subsubsection*{Tensiometers} Tensiometers measure soil water
tension. An air-free tube of water is attached to a porous tip, such as
ceramic. As water extracted from the soil by plants, the vacuum in the
tube increases. When the soil is irrigated, the pressure decreases. A
pressure sensor at the top of the tube is used to make the moisture
measurements. The response time to moisture changes is typically 2-3
hours. Tensiometers are slightly cheaper than the volumetric sensors
discussed earlier, but still require a dedicated data
logger/transmitter. They typically require regular maintenance, such as
adding water or removing air from the tube. Tensiometers also only
operate when soils are relatively wet, and cannot be used when there
is any risk of freezing temperatures.

\subsubsection*{Resistive sensors}
Resistive sensors are among the cheapest sensors. These sensors have
two probes in the soil and current passes through the soil between the
probes. The moisture value is determined by measuring the
resistance. DIY soil moisture kits, like the SparkFun~\cite{sparkfun}
sensor, are usually resistive. Resistance-based sensors are the least
accurate, and the probes corrode quickly~\cite{capVSres}. Granular
matrix sensors (GMS) are a variation on resistive sensors. They
measure electrical resistance in a porous medium like ceramic. As
water seeps into the porous medium, it reduces the electrical
resistance. The ceramic in GMSes prevents probe decay and naturally
filter out ions from fertilizer and salt, which improves the accuracy
of the readings. GMSes cost similar to tensiometers, but take longer
to respond, do not work well in very wet soil, and are more difficult
to install. They do, however, require significantly less maintenance.

\subsubsection*{Volumetric sensors}Volumetric sensors estimate
$\frac{V_{water}}{V_{wet soil}}$, the ratio of the volume of the water
to the volume of the soil plus the same volume of water. As the
volumetric water content changes, the dielectric permittivity constant
of the soil changes. Recall that dielectric permittivity,
$\varepsilon$, is the ability of a substance to hold an electrical
charge. The dielectric permittivity constant (also known as relative
permittivity), $\varepsilon_r$, is the ratio of the permittivity of
the substance to the permittivity of free space,
$\varepsilon_0$. Permittivity is often treated as a complex number:
$\varepsilon = \varepsilon' + j\varepsilon ''$, where $\varepsilon '$
is the real component and $\varepsilon''$ is the complex.

There are two main types of volumetric sensors: capacitive and time
domain refelectometry (TDR). Capacitive sensors measure the charge
time of a capacitor, which is a roughly linear function of
$\varepsilon$~\cite{sensorOverview}. Capacitive sensors are much less
prone to corrosion than resistive sensors, and are more accurate. For
this reason, capacitive sensors are available as commercial-grade soil
moisture sensors. 

Time domain reflectometry (TDR) is another common method for measuring
soil moisture that takes advantage of the frequency dependence of the
dielectric permittivity. It measures the propagation time of EM waves
by sending a pulse down a cable and into the soil probe and measuring
how long it takes before the signal is reflected and returns. We call
this the \emph{time of flight}, or ToF. This method of soil moisture
measurement generates wideband signals (100Mhz-3Ghz for agricultural
grade) to ensure sufficient time resolution. Like capacitive sensors,
the probes must maintain good contact with the soil for accurate
measurements. The cost of a sensor is about \$1000.

A related technique uses ground penetrating radar (GPR) to calculate
the time of flight. GPR is also wideband and costs thousands of
dollars. Generally the equipment is large (~the size of a lawnmower)
and needs to be dragged across the surface of the soil. This is a
labor intensive process which may not always be possible in dense
crops. Non-contact GPRs exist that can be attached to drones, but
these have lower resolution and can only reliably measure to a depth
of about 10cm~\cite{gpr}.

\subsubsection*{Novel soil moisture sensors}
A number of recent efforts have explored ways to lower the cost of
soil moisture sensors. \cite{Hasan2013} and uses passive RFID tags to sense
impedance changes caused by soil moisture, but their system requires a
\$1000+ RFID reader, a free-air reference tag, and a delicate antenna
needs to remain on the surface of the soil in order to collect the
readings. \cite{Dey2016} measures the shift in resonance frequency of the
scattering parameter (S-parameter). Their technique required the use
of a vector network analyzer (VNA), so it would require a potentially
costly custom reader. Both of the above approaches were only tested in
an indoors laboratory environment, and they do not provide any
comparisons to the 'ground truth' or an existing commercial sensor.

\cite{Daskalakis2016} attaches a capacitive sensor to an above-ground dipole
antenna. The readings are collected using a low-cost software defined
reader. The authors claim a cost of 6 Euro, which is the lowest-cost
sensor that was actually deployed. However, this cost is without any
weatherproofing, and a key tradeoff is a 4x decrease in resolution
over commercial sensors.


In 2017 researchers at Microsoft Research figured out how to measure
soil moisture using consumer-grade WiFi signals. This system, called
SMURF~\cite{smurf}, uses time of flight like TDR and GPR. However, in
order to use the relatively narrow bandwidth of WiFi it relies on the
\emph{relative} time of flight, which is the difference in time of
arrival between antennas in a MIMO system. The system is still
difficult to deploy and maintain because it requires burying multiple
WiFi antennas in the ground and connecting them via cable to an active
receiver such as an Atheros WiFi card. Then, a different active
transmitter must send packets to the antennas in the ground in order
to collect the measurements. The antennas are not as sensitive to soil
contact as TDR and capacitive sensors, but the soil packing between
antennas impacts the measurements. While SMURF is an important
stepping stone, this work is impractical for deploying at-scale on a
farm.

\section*{Proposed approach}
How can we take the low-cost of SMURF and make it practical to deploy
and use? The extremely low cost of backscatter tags (such as RFID) make
it an attractive option to explore. Passive RFID tags can be purchased
for as little as 7 cents~\cite{rfidcost}. The need to use a
proprietary reader is eliminated with WiFi backscatter tags
like~\cite{Zhang2017}~\cite{Zhang2016}. When the cost of the sensors
is so low, the system becomes more maintainable because it is
inexpensive to replace failing or broken sensors. Furthermore, the
small form factor of backscatter tags makes them easier to embed in
the soil, which makes it possible to take measurements at multiple
depths per sensor site. It also opens up opportunities for novel
deployment systems, such as attaching the tags to a soil auger.

The same relative time of flight approach cannot be used, though,
since the power savings and low cost of backscatter tags comes from
the lack of an active receiver. Calculating accurate ToF from
commodity WiFi devices is very difficult~\cite{Vasisht2015}, which is
why SMURF uses relative ToF. It is potentially possible for a MIMO
receiver to separate signals from multiple tags using the MUSIC
algorithm~\cite{Soltanaghaei2018}~\cite{Kotaru2015} and calculate the
ToF delta between the multiple reflections, but significant challenges
still remain to to get enough signal strength from the buried
backscatter devices.

WiFi is attractive because of its ubiquity and low cost, but
consumer-grade RADAR systems are becoming increasingly affordable and
accessible, which introduces exciting new sensing possibilities. There
are multiple radar devices on the market compatible with Android
phones~\cite{walabot}~\cite{omnipresense}, ranging from \$50-200. On their own,
consumer-grade radar systems are incapable of sensing soil
moisture---that would require the still-expensive technology of ground
penetrating radar. These cheap radars, however, could be paired with
backscatter tags.

\begin{figure*}[h!]
  \centering
  \includegraphics[scale=0.4]{soil_moisture_diagram.png}\\
  \caption{A potential deployment of a UWB RADAR backscatter soil
    moisture sensing system, where the radar is connected to the
    farmer's smartphone}
\end{figure*}

Consumer-grade ultra-wideband Radars have the ability to accurately
measure the ToF of reflected signals. UWB radars typically have a
bandwidth of 3-7Ghz. This corresponds to a time resolution of
0.15-0.33ns. SMURF requires about 0.1ns ToF resolution. For
comparison, agricultural TDR sensors range from 100Mhz to
3Ghz~\cite{Pelletier2012} which leads to a resolution of at most
0.33ns. The ToF for a ranging radar is calculated by converting the
incoming samples back to the time domain and measuring the time delta
between peaks. Thus the sharper the peaks, the finer the possible
time resolution. A radar-based approach could potentially have
improved ToF resolution by using the phase of the reflected signal in
addition to the usual ranging estimate technique. The time delta
between the peaks would be a coarse estimate, and the phase of the
signal a fine tuning.

% The high bandwidth of the radars allows us to do a CDMA-inspired
% process introduced in ~\cite{}.

There are no commercially available backscatter tags currently
designed for use with radar, but fortunately the open-source design
for the HitchHike WiFi backscatter tags are compatible once the RF
switch is changed to match the bandwidth that the radar operates on.

One the tags are confirmed to be working with the radar above-ground,
the next key challenge will be burying the tags and discerning the
reflected backscatter signal from reflections due to the soil. One
possible approach is to toggle the backscatter tag at a relatively low
frequency, 100-500Hz. To the radar, this makes the tag look like it is
in motion, which makes the backscatter signal easy to pick out when
the rest of the environment is stationary. This clever approach was
designed by a labmate and is under review with no preprint. It is discussed in detail in an appendix.

The ability to isolate the backscatter signal from other reflections
also has implications for how far the tags can be buried. The ideal
maximum depth would be 1+ meters, as crops that grow on trees often
benefit from moisture measurements at that depth. Depths of 10-30cm
are more than sufficient for a number of non-tree crops, though.

For the radar, a few low cost options are available. One is the
Walabot Developer UWB pulsed radar, which is readily commercially
available and has a bandwidth of 6.7Ghz. This corresponds to a ToF
resolution of 0.15ns. ~\cite{uwbThesis} shows that it is well suited
to being mounted on drones or autonomous robots for automated data
collection. This would allow both small-holder farmers and large
commercial farms to benefit. However, the frame rate supported by the
SDK may be too low. The documentation is sparse, so experiments are
needed.

The XeThru X4 radar series is another good option. The frame rate is
definitely sufficient to isolate a backscatter toggle between
100-500hz, but the bandwidth is more narrow at 2.5Ghz. 

\subsection*{}

\section*{Next steps}
\begin{itemize}
\item Now---early January: work with open-air tags and radar SDKs to get
  good ToF results, recruit student collaborators to assist with the
  labor-intensive measurements once soil experiments begin
\item Mid January---early February: embed the tags in soil and conduct
  maximum soil depth experiments, confirming that the backscatter
  signal can be discerned from reflections due to the soil
\item February: conduct soil moisture experiments by measuring
  permittivity and comparing against commercial sensor (in different
  soil types if possible). Also quantify and improve tag battery life
  (if applicable).
\item March: wrap up results and write the paper, March 12 is the
  Mobicom 2019 deadline
\end{itemize}

\section*{Future work}
\subsubsection*{Electrical Conductivity} It might also be possible for a
backscatter system to measure electrical conductivity (EC), which is
another parameter that soil moisture sensors sometimes measure. EC is
related to the salinity of the soil and is correlated strongly with
crop health.

\subsubsection*{Calibration} One current weakness with all sensors is the
need to calibrate them to specific soil types. Is it possible to set
easy-to-follow guidelines, or even to automatically detect and
calibrate for the soil type? This would massively decrease the
deployment and usage barriers.

\subsubsection*{Environmental impact of tags}
How readily do the tags get detached and lost in the soil? How does
this impact the growing environment, especially if the tags use
batteries?

\subsubsection*{Leaf moisture}
Leaf moisture or leaf thickness is a more reliable measure of how much
water a plant needs, but it is significantly more difficult to collect
than soil moisture. The sensors without data loggers and
communications are \$300 and are unwieldy and
delicate. ~\cite{Wang2017} uses RFID tags to identify and measure
liquids, and can even tell the difference between a vessel containing
Coke and Pepsi. Could this be extended to measuring the leaf thickness
or nutrient absorption of plants?


\section*{Conclusion}
Technology in agriculture is a relatively unexplored area, especially
within the academic EE and communications communities. This creates
opportunities for cross-disciplinary work and opens up a rich set of
open problems with global impact. Using RF backscatter to detect
soil moisture is an interesting and useful application of backscatter
and radar technologies, with multiple potential future works. We
believe it will address the shortcomings of present sensing solutions,
and will also be accessible to small and large farms alike.

\bibliographystyle{amsplain}
\begin{footnotesize}
  \bibliography{soil_proposal}
\end{footnotesize}

\section*{Appendix: Radar}
RADAR (RAdio Detection And Ranging) is a well-developed technology
with a history dating back to before World War II. Conceptually
similar to echolocation, it uses the Time of Flight and angle of
arrival of RF waves to calculate the distance of objects and their
speed.

Radar was originally used primarily in military contexts, but nowadays
it is used to predict weather patterns, measure the speed of vehicles
and other objects, assist autonomous vehicle navigation, and even
monitoring human breathing.

There are two primary types of radar waveforms: continuous wave and
pulsed~\cite{Richards2010}. Continuous wave radars are transmitting
and receiving at all times, while pulsed radars periodically transmit a
short-duration pulse and listen for the reflections to come back.

The most basic types of continuous wave radar use the Doppler shift of
moving targets to calculate their speed, and they are incapable of
calculating the target distance, or \emph{range}. Frequency Modulated
Continuous Wave (FMCW) is a variant that is capable of calculating
range as well as speed. It periodically sweeps the transmit signal
through a frequency range, and measures the frequency delta between
the transmitted and received signal to calculate distance.

Pulsed radars\footnote{Pulse radars are also sometimes called
  pulse-Doppler radars because they Doppler to measure the speed of
  targets. Confusingly, basic CW radars are sometimes just called
  Doppler radars because they too use Doppler to measure speed.} can
very easily determine the range of a target by measuring the time that
elapses between pulse transmission and the return reflection. They can
also measure target speed. When the pulse width of the radar is very
short, it is known as an impulse radar or Ultra Wideband (UWB)
radar. UWB radars were enabled by a key circuits discovery in 1994:
the single-shot transient digitizer~\cite{}. This device is capable of
high speed, high accuracy digitization of very short pulses of enegry
(< 5ns). This allowed for the construction of a significantly cheaper
and smaller radar transceiver that digitizes incoming RF and
correlates it with the transmitted pulse samples. 

Impulse radar needs to be wideband because of Fourier duality:
pulses that are short in the time domain require wide bandwidth in the
frequency domain. The bandwidth of a UWB radar typically ranges
between 2 and 8 Ghz. Furthermore, the transmit power of UWB radars is
usually regulated to be very low to avoid causing interference to
other users on the same spectrum. This also makes UWB radar very
difficult to detect, as the transmitted waveform looks like white
noise. UWB radars are also popular because the low transmission power
ensures that the signal is harmless to humans. The transmit power
regulations also restrict the operating range of UWB radars to
relatively short distances, typically $<25$m for consumer grade
radars.

A modern radar operates using \emph{frames}, which is the samples from
a single sweep across the radar's sensing area. The array of data
returned has one complex sample per \emph{range bin}. Each range bin
corresponds to a range of possible distances from the radar. For
example, if the radar resolution is 5cm, the magnitude of the sample
from the 10th range bin would correspond to an object that is 45-50cm
away from the radar.

Let the transmitted impulse be represented by $p(t)$, and the received
frame represented by $f(n)$. The digitized frame can be written as

\begin{equation}
f[n] = \sum\limits_{k=1}^K \alpha p(nT - 2d_k/c)  
\end{equation}

The \emph{dynamic range} of a radar...We need to find the dynamic
range of the radar (ADC bits)

The \emph{leakage} of a radar from TX to RX...

To figure out roughly how far we can go under ground, we need the
signal from the tag to be greater than the leakage minus the dynamic
range...


\section*{Soil equations}
Time of flight, $\tau$, meaures the \emph{apparent dielectric
  constant} $K_a$, which is a function of
$\varepsilon_r', \varepsilon_r'', \varepsilon_0$ and electical conductivity (EC)
$\sigma$:

\begin{equation}
  K_a = \frac{\varepsilon_r'}{2}\Bigg[\sqrt{1+\bigg(\frac{\varepsilon_r'' + \frac{\sigma}{2\pi f\varepsilon_0}}{\varepsilon_r'}\bigg)^2}+1\Bigg]  
\end{equation}

At high frequencies, $\epsilon_r$ is dominated by the real part
$\epsilon_r'$, so $K_a$ approximates $\epsilon_r'$. Then we have

\begin{equation}
K_a \approx (\frac{c\tau}{d})^2
\end{equation}

where $c$ is the speed of light and $d$ is the known distance (in
meters) that we calculate the ToF for.

Soil moisture $\Theta$ can be related to $K_a$ using formulas that depend on the soil type. One such formula is the Topp Equation ~\cite{Topp1980}, which is applicable to ``average soils'':

\begin{equation}
  \Theta = 4.3\times 10^{-6}K_a^3-5.5\times10^{-4}K_a^2+2.92\times 10^{-2}K_a-5.3\times 10^{-2}
\end{equation}

Where the units of soil moisture are $cm^3/cm^3$. This the ratio of water volume over dry soil volume.

Substituting,

\begin{equation}
  \Theta \approx 4.3\times 10^{-6} (\frac{c\tau}{d})^6-5.5\times10^{-4} (\frac{c\tau}{d})^4+2.92\times 10^{-2} (\frac{c\tau}{d})^2-5.3\times 10^{-2}
\end{equation}

To be on par with commercial soil sensors, we want our soil moisture
measurements to be within $0.05 cm^3/cm^3$ of the actual soil moisture
(measured with an oven). This also means we need a resolution of at
least $0.05 cm^3/cm^3$.

Waves slow down 2-6 times in soil compared to the speed of
light~\cite{gpr}. Assuming a distance of 1m and a slowdown of 3x,
$\tau = 10.006$ns. Then,

\begin{equation}
  \Theta \approx 3.1\times 10^{45}\tau^6-4.44\times 10^{30}\tau^4+2.6\times 10^{15}\tau^2-0.053 = 0.1657
\end{equation}

Referring to Table 1, With a ToF resolution of 1ns, we will get a
delta of $0.035-0.04 cm^3/cm^3$, which is within our bounds. However,
the equation is nonlinear, so a smaller ToF delta is helpful. If we
achieve 0.1ns, the resulting VWC resolution is about 0.003-0.004. Of
course, just because we have the necessary resolution does not mean
our sensor will be accurate within 0.05$cm^3/cm^3$.


\begin{table}[]
  \centering
\begin{tabular}{l|l}
\textbf{ToF (ns)} & \textbf{VWC ($cm^3/cm^3$)} \\ \hline
9.0               & 0.1301                                                      \\
9.9              & 0.1620                                                       \\
  10.0              & 0.1657                                                       \\
10.1              & 0.1693                                                       \\
11.0              & 0.2020                                                      
\end{tabular}
\end{table}

The next key question is, how do we pick out the weak backscatter
signal against other reflections?



\end{document}
